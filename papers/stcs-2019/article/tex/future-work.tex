\documentclass[main.tex]{subfiles}
\begin{document}
	
	Currently, the framework only supports the compilation of Idris programs through the Idris front end, but we are working on supporting Haskell by integrating the Glasgow Haskell Compiler as a new front end. As of right now, the framework \emph{can} generate GRIN IR code from GHC's STG representation, but the generated programs still contain unimplemented primitive operations. The main challenge is to somehow handle these primitive operations. In fact, there is only a small set of primitive operations that cannot be trivially incorporated into the framework, but these might even require extending the GRIN IR with additional built-in instructions.
	
	Besides the addition of built-in instructions, the GRIN intermediate representation can be improved further by introducing the notion of function pointers and basic blocks. Firstly, the original specification of GRIN does not support modular compilation. However, extending the IR with function pointers can help to achieve incremental compilation. Each module could be compiled separately with indirect calls to other modules through function pointers, then by using different data-flow analyses and program transformations, all modules could be optimized together incrementally. In theory, if the entire program is available for analysis at compile time, incremental compilation could produce the same result as whole program compilation. Secondly, the original GRIN IR has a monadic structure which can make it difficult to analyze and transform the control flow of the program. Fortunately, replacing the monadic structure with basic blocks can resolve this issue.
	
  Whole program analysis is a powerful tool for optimizing compilers, but it can be quite demanding on execution time. This being said, there are certain techniques to speed up these analyses. The core of the GRIN optimizer is the heap points-to analysis, an Andersen-style inclusion based pointer analysis~\cite{andersen-ptr}. This type of data-flow analysis is very well researched, and there are several ways to improve the algorithm's performance. Firstly, cyclic references could be detected and eliminated between data-flow nodes at runtime. This optimization allows the algorithm to analyze millions of lines of code within seconds~\cite{andersen-opt}. Secondly, the algorithm itself could be parallelized for both CPU and GPU~\cite{andersen-gpu}, achieving considerable speedups. Furthermore, some alternative algorithms could also be considered. For example, Steengaard's unification based algorithm~\cite{steensgaard-ptr} is a less precise analysis, but it runs in almost linear time. It could be used as a preliminary analysis for some simple transformations at the beginning of the pipeline. Finally, Shapiro's algorithm~\cite{shapiro-ptr} could act as a compromise between Steengaard's and Andersen's algorithm. In a way, Shapiro's analysis lies somewhere between the other two analyses. It is slower than Steengaard's, but also much more precise; and it is less precise than Andersen's, but also much faster.
  
  All things considered, the current implementation of GRIN brought adequate results. However, there are still many promising ideas left to research.
	
\end{document}